\documentclass[11pt]{article}
\usepackage[utf8]{inputenc}
\usepackage{geometry}
\geometry{a4paper, margin=1in}
\usepackage{enumitem}
\usepackage{amsmath}
\title{BTP Progress Log: Microswimmer Bimotility Simulation}
\author{Prince Evans  \& Priyanshu Deora \\ \small Supervisor: Prof. Sujin B. Babu}
\date{December 2025 - April 2026}

\begin{document}

\maketitle

\section*{Week 1: Project Initiation \& Literature Review}
\textbf{Interval:} December 15, 2025 -- December 21, 2025

\begin{itemize}[noitemsep]
    \item \textbf{Paper Analysis:} Reviewed "Self-organization in a bimotility mixture of model microswimmers" by Adyant Agrawal and Sujin B. Babu[cite: 3, 4].
    \item \textbf{Model Definition:} Identified the Taylor line as a discretized model of the Taylor sheet for microswimmer simulation[cite: 20, 36].
    \item \textbf{Research Objective:} To study cooperation and segregation in mixtures differing in propulsion speed[cite: 8, 32].
    \item \textbf{Task Distribution:}
    \begin{itemize}
        \item \textbf{Partner A:} Environment physics (Circular boundary and Steric interactions).
        \item \textbf{Partner B:} Swimmer internals (Chain physics and Propulsion engine).
    \end{itemize}
\end{itemize}

\section*{Week 2: Implementation of Dry Simulation}
\textbf{Interval:} December 22, 2025 -- December 28, 2025

\subsection*{Partner A: Boundary \& Steric Interaction}
\begin{itemize}[noitemsep]
    \item \textbf{Confinement Logic:} Implemented rigid circular boundary conditions as per paper specifications[cite: 49, 169].
    \item \textbf{Reflection Algorithm:} Coded velocity reflection along the normal vector for bead-boundary collisions.
    \item \textbf{Steric Force:} Implemented the truncated Lennard-Jones potential[cite: 62, 63]:
    \[ V_{l}=4\epsilon\left[\left(\frac{r_{o}}{r}\right)^{12}-\left(\frac{r_{o}}{r}\right)^{6}\right] \text{ for  }  r < 2^{1/6}r_o \]
    \item \textbf{Interaction Parameters:} Set $\epsilon = 13.75$ and $r_o = a_0$ to model swimmer-swimmer repulsion[cite: 66].
\end{itemize}

\subsection*{Partner B: Taylor Line Internal Physics}
\begin{itemize}[noitemsep]
    \item \textbf{Swimmer Representation:} Each swimmer is modeled as a discretized Taylor line composed of beads connected in sequence, capturing undulatory microswimmer motion at low Reynolds number.
    \item \textbf{Elastic Spring Forces:} Neighboring beads interact via Hookean springs
    \[
    \mathbf{F}_s = k_s(r - l_0)\hat{\mathbf{r}}
    \]
    ensuring inextensibility and structural integrity of the swimmer.
    \item \textbf{Active Bending Mechanism:} A traveling sinusoidal curvature wave is imposed along the body to generate propulsion, given by
    \[
    c(i, t) = b \sin\left[2\pi\left(\nu t + \frac{2i}{N}\right) + \phi\right]
    \]
    \item \textbf{Bending Force Formulation:} Discrete curvature
    \[
    \mathbf{C}_i = \mathbf{r}_{i+1} - 2\mathbf{r}_i + \mathbf{r}_{i-1}
    \]
    produces bending forces
    \[
    \mathbf{F}_b = \kappa \, c(i, t) \, \mathbf{C}_i
    \]
    \item \textbf{Overdamped Dynamics:} Bead motion follows viscous-dominated dynamics
    \[
    \gamma \mathbf{v}_i = \mathbf{F}_i, \quad \mathbf{r}_i(t + \Delta t) = \mathbf{r}_i(t) + \mathbf{v}_i \Delta t
    \]
    \item \textbf{Outcome:} The implemented internal physics yields smooth, force-free, self-propelled Taylor-line swimmers consistent with theoretical and experimental microswimmer models.
\end{itemize}

\end{document}